% RESUMO--------------------------------------------------------------------------------

\begin{resumo}[RESUMO]
\begin{SingleSpacing}

% Não altere esta seção do texto--------------------------------------------------------
\imprimirautorcitacao. \imprimirtitulo. \imprimirdata. \pageref {LastPage} f. \imprimirprojeto\ – \imprimirprograma, \imprimirinstituicao. \imprimirlocal, \imprimirdata.\\
%---------------------------------------------------------------------------------------

Nesse relatório de estágio, requisito para conclusão do Curso Superior de Tecnologia em Sistemas para Internet, são apresentadas atividades referentes ao planejamento, desenvolvimento e execução do Sistema de Gestão de Frotas denominado Veículos, para o Tribunal Regional Eleitoral da Paraíba. O projeto tem previsão para ser concluído em até seis meses, e será supervisionado pelo chefe da SEDES, Francisco Gomes, bem como acompanhado pelo supervisor técnico, que é o desenvolvedor responsável pelo projeto, efetivo do Tribunal. A equipe no projeto fica então composta por 1 gerente e 3 desenvolvedores, sendo 1 supervisor técnico e 2 estagiários, 1 Product Owner e 1 DBA responsável pelo gerenciamento do banco de dados, este lotado na SISBAN. Esse projeto colaborou profundamente para a fixação dos conhecimentos adquiridos no decorrer do curso, principalmente nas tecnologias:Java, JSF, SQL e SVN. O principal objetivo do estágio foi utilizar, na prática, os conhecimentos obtidos ao longo do curso, bem como aprender conceitos e tecnologias extras, adquirir experiência com a rotina de trabalho e principalmente aprender a codificar em equipe utilizando versionamento de código, seguindo todas as etapas previstas no PDS denominado MODUS.\\

% O Resumo é um elemento obrigatório em tese, dissertação, monografia e TCC, constituído de uma seqüência de frases concisas e objetivas, fornecendo uma visão rápida e clara do conteúdo do estudo. O texto deverá conter no máximo 500 palavras e ser antecedido
% pela referência do estudo. Também, não deve conter citações. O resumo deve ser redigido em parágrafo único, espaçamento simples e seguido das palavras representativas do conteúdo do estudo, isto é, palavras-chave, em número de três a cinco, separadas entre si por ponto e finalizadas também por ponto. Usar o verbo na terceira pessoa do singular, com linguagem impessoal, bem como fazer uso, preferencialmente, da voz ativa. Texto contendo um único parágrafo.\\

\textbf{Palavras-chave}: Java. JSF. SQL. Gestão. Desenvolvimento.

\end{SingleSpacing}
\end{resumo}

% OBSERVAÇÕES---------------------------------------------------------------------------
% Altere o texto inserindo o Resumo do seu trabalho.
% Escolha de 3 a 5 palavras ou termos que descrevam bem o seu trabalho 

