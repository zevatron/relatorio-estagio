% ABSTRACT--------------------------------------------------------------------------------

\begin{resumo}[ABSTRACT]
\begin{SingleSpacing}

% Não altere esta seção do texto--------------------------------------------------------
\imprimirautorcitacao. \imprimirtitleabstract. \imprimirdata. \pageref {LastPage} f. \imprimirprojeto\ – \imprimirprograma, \imprimirinstituicao. \imprimirlocal, \imprimirdata.\\
%---------------------------------------------------------------------------------------

Elemento obrigatório em tese, dissertação, monografia e TCC. É a versão do resumo em português para o idioma de divulgação internacional. Deve ser antecedido pela referência do estudo. Deve aparecer em folha distinta do resumo em língua portuguesa e seguido das palavras representativas do conteúdo do estudo, isto é, das palavras-chave. Sugere-se a elaboração do resumo (Abstract) e das palavras-chave (Keywords) em inglês; para resumos em outras línguas, que não o inglês, consultar o departamento / curso de origem.\\

\textbf{Keywords}: Java. JEE. JSF, Gerenciamento, Projeto, Veículos.

\end{SingleSpacing}
\end{resumo}

% OBSERVAÇÕES---------------------------------------------------------------------------
% Altere o texto inserindo o Abstract do seu trabalho.
% Escolha de 3 a 5 palavras ou termos que descrevam bem o seu trabalho 
