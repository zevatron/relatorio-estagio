% INTRODUÇÃO-------------------------------------------------------------------

\chapter{Introdução}
\label{chap:introducao}

% Edite e coloque aqui o seu texto de introdução.

% A Introdução é a parte inicial do texto, na qual devem constar o tema e a delimitação do assunto tratado, objetivos da pesquisa e outros elementos necessários para situar o tema do trabalho, tais como: justificativa, procedimentos metodológicos (classificação inicial), embasamento teórico (principais bases sintetizadas) e estrutura do trabalho, tratados de forma sucinta. Recursos utilizados e cronograma são incluídos quando necessário. Salienta-se que os procedimentos metodológicos e o embasamento teórico são tratados, posteriormente, em capítulos próprios e com a profundidade necessária ao trabalho de pesquisa.
Este relatório descreve as atividades realizadas, no que diz respeito ao planejamento, desenvolvimento e execução de um dos projetos desenvolvidos pela SEDES, denominado de Veículos – Sistema de gestão de frotas do TRE-PB. O mesmo foi executado no exercício do estágio supervisionado no Tribunal Regional Eleitoral da Paraíba, localizado no Centro em João Pessoa.
Serão explanados, tanto os conhecimentos teóricos, quanto os práticos, descrevendo todas as etapas do projeto. Também estarão presentes neste relatório informações referentes à empresa e sobre a unidade da empresa onde o projeto foi executado, além de informações sobre infra-estrutura.
Por fim, serão mencionadas as correlações entre o conteúdo visto em sala de aula e o que foi feito na prática, bem como as principais dificuldades encontradas durante a execução do projeto.

% \section{LEIA ESTA SEÇÃO ANTES DE COMEÇAR}
% \label{sec:antesleiame}

% Este documento é um \emph{template} \LaTeX{} que foi concebido, primariamente, para ser utilizado na elaboração de Trabalho de Conclusão de Curs em conformidade com as normas da Universidade Tecnológica Federal do Paraná.

% Para a produção deste \emph{template} foi necessário adaptar o arquivo {\ttfamily abntex2.cls}. Assim, foi produzido o arquivo {\ttfamily utfpr-abntex2.cls} que define o \verb|documentclass| específico para a UTFPR.

% Antes de começar a escrever o seu trabalho acadêmico utilizando este \emph{template}, é importante saber que há dois arquivos que você precisará editar para que a capa e a folha de rosto de seu trabalho sejam geradas automaticamente.
% São eles os arquivos {\ttfamily capa.tex} e {\ttfamily folha-rosto.tex}, ambos no diretório  {\ttfamily /elementos-pre-textuais}.
% No arquivo {\ttfamily capa.tex} deverá ser informado nome do autor, título do trabalho, natureza do trabalho, nome do orientador e outras informações necessárias.
% No arquivo {\ttfamily folha-rosto.tex}, que contém o texto padrão estabelecendo que este documento é um requisito parcial para a obtenção do título pretendido, será necessário apenas comentar as linhas que não se aplicam ao tipo de trabalho acadêmico.

% A compilação para gerar um arquivo no formato pdf, incluindo corretamente as referências bibliográficas, deve ser realizada utilizando o comando \verb|makefile|, disponível na mesma pasta onde está o arquivo principal \verb|utfpr-tcc.tex|. Caso seja alterado o nome do arquivo \verb|utfpr-tcc.tex|, deverá ser alterado no arquivo \verb|makefile| também.

\section{Objetivo}
\label{sec:objetivo}

\subsection{Objetivo geral}
\label{sec:objetivoGeral}
Auxiliar no desenvolvimento dos ativos, soluções web desenvolvidas pela SEDES, no TRE-PB.  Para vistas deste relatório será utilizado como referência o projeto \imprimirtitulo. O sistema deverá ser capaz de receber solicitações dos usuários e exibi-las em um painel onde o gestor deverá analisar os pedidos e montar as viagens adequando: rotas, disponibilidades dos motoristas e passageiros. Nas atividades são contempladas todas as etapas descritas no Modelo de Desenvolvimento de Software adotado pelo TRE-PB.

\subsection{Objetivos específicos}
\label{sec:objetivosEspecificos}
Contribuir na implementação do Sistema de Gestão de Frotas, quando possível oferecendo alternativas para a implementação mediante aos conhecimentos adquiridos na graduação. Ganhar experiência com o fluxo de trabalho da uma equipe de desenvolvimento incorporando conceitos de versionamento de código, sendo capaz de resolver conflitos de códigos, quando houver, e evita-los. Trabalhar com reuniões diárias utilizadas pela abordagem da metodologia SCRUM, planejamento das soluções com a modelagem do negócio e com entregas frequentes sempre buscando feedback das releases pelo cliente. 


% \section{ORGANIZAÇÃO DO TRABALHO}
% \label{sec:organizacaoTrabalho}

% Normalmente ao final da introdução é apresentada, em um ou dois parágrafos curtos, a organização do restante do trabalho acadêmico.
% Deve-se dizer o quê será apresentado em cada um dos demais capítulos.

\section{A Empresa}\footnote{http://www.tre-pb.jus.br/institucional/conheca-o-tre-pb/conheca-o-tre-pb}
\label{sec:empresa}
A Justiça Eleitoral é o ramo especializado do Poder Judiciário que visa garantir a lisura, a eficiência e a eficácia do processo eleitoral, contribuindo para o fortalecimento da democracia e a consolidação do Estado de Direito. Compete à Justiça Eleitoral preparar, realizar e apurar as eleições, além de administrar o Cadastro Nacional de Eleitores.
O principal objetivo da Justiça Eleitoral é o gerenciamento do processo eleitoral, através de diretrizes claras e firmes, evitando vícios, abusos e fraudes.
O Tribunal Regional Eleitoral da Paraíba - TRE-PB, órgão máximo da Justiça Eleitoral no Estado, tem como instância superior, em matéria eleitoral, o Tribunal Superior Eleitoral, sediado em Brasília - Distrito Federal. A finalidade do TRE-PB é planejar e coordenar o processo eleitoral nas eleições federais, estaduais e municipais, no âmbito do Estado da Paraíba.
Compete, também, ao Tribunal, julgar os recursos interpostos das decisões dos Juízes e Juntas Eleitorais do Estado, bem como, os processos originários e administrativos do próprio Tribunal; registrar os partidos e candidatos a cargos eletivos de Governador, Senador, Deputado Federal e Estadual, assim como, receber e analisar a prestação de contas dos mesmos, prestadas ao final de cada campanha estadual; analisar as prestações de contas anuais dos órgãos regionais dos partidos políticos; elaborar e fiscalizar o calendário estadual de propaganda eleitoral; proceder à anotação e cancelamento dos diretórios estaduais e municipais dos partidos políticos; julgar as impugnações relativas aos pedidos de registros de candidaturas e as arguições de inelegibilidade; designar os Juízes Titulares das Zonas Eleitorais do Estado da Paraíba e administrar o Cadastro de Eleitores.

\section{Descrição geral das atividades}
\label{sec:descricaoGeralAtividades}
Durante o processo foram utilizadas as metodologias Scrum para gestão e planejamento dos projetos e o Kanban para o controle de fluxo do desenvolvimento. As atividades desenvolvidas no período do estágio foram as seguintes de acordo com as fases: 
\begin{enumerate}
    \item Imersão:
    \begin{enumerate}
        \item Elicitação das histórias de usuário.
        \item Definição do escopo do produto.
        \item Criar estrutura do projeto.
        \item Modelagem de dados.
    \end{enumerate}
   \item Construção:
   \begin{enumerate}
        \item Refinar histórias de usuário.
        \item Estimar histórias.
        \item Codificação de funcionalidades.
        \item Preparar versão para homologação.
        \item Codificar ajustes.
        \item Gerar documentação.
        \item Implantar versão.
   \end{enumerate}
\end{enumerate}

\section{Organização do relatório}
\label{sec:organizacaoRelatorio}
Além desse capítulo, o relatório está dividido em outros quatro capítulos brevemente descritos abaixo:

Capítulo 2 - Embasamento Teórico: Aborda as tecnologias e linguagens que foram utilizadas durante o estágio, bem como as definições necessárias para a compreensão do processo.

Capítulo 3 - \imprimirtitulo: apresenta o projeto, no qual as atividades do estagiário foram realizadas, descrevendo os principais conceitos, arquitetura e funcionalidades.

Capítulo 4 – Atividades Realizadas: Relata as atividades realizadas ao longo do período de estágio; descreve o fluxo e o processo de desenvolvimento dessas atividades e detalha a implementação das funcionalidades.

Capítulo 5 – Conclusão: São feitas as considerações finais, ou seja, um relato sobre as experiências adquiridas, metas e objetivos alcançados, o que poderia ter sido feito para obter um melhor resultado final, erros cometidos e trabalhos futuros.
