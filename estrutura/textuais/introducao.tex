% INTRODUÇÃO-------------------------------------------------------------------

\chapter{Introdução}
\label{chap:introducao}

Este relatório descreve as atividades realizadas, no que diz respeito ao planejamento, desenvolvimento e execução de um dos projetos desenvolvidos pela SEDES, denominado de Veículos – Sistema de gestão de frotas do TRE-PB. O mesmo foi executado no exercício do estágio supervisionado no Tribunal Regional Eleitoral da Paraíba, localizado no Centro em João Pessoa.

O projeto teve início em abril de 2017 e a versão 1.0 foi entregue na metade de maio de 2017 e continha cadastros iniciais como colaboradores e veículos. A versão 1.0.1 se estendeu até julho de 2017 e continha as principais funcionalidades da aplicação que de fato iriam realizar o controle das viagens. O projeto tinha previsão para ser concluído em dois meses, porém uma mudança substancial do modelo teve que ser efetuada durante a fase de desenvolvimento gerando a versão 1.1, como pode ser visto na ata de reunião descrita na \autoref{fig:figura-ata1}. Dessa forma o projeto precisou ser replanejado e se estendeu, totalizando cinco meses. Ele foi supervisionado pelo chefe da SEDES, Francisco Gomes, bem como acompanhado pelo supervisor técnico, que é o desenvolvedor responsável pelo projeto que também precisou ser substituído, todos efetivos no quadro do Tribunal. A equipe no projeto ficou composta por 1 gerente e 3 desenvolvedores, sendo 1 supervisor técnico e 2 estagiários, 1 Product Owner e 1 DBA responsável pelo gerenciamento do banco de dados, este lotado na SISBAN.

Nesse relatório serão explanados, tanto os conhecimentos teóricos, quanto os práticos, descrevendo todas as etapas do projeto. Também estarão presentes neste relatório informações referentes à empresa e sobre a unidade da empresa onde o projeto foi executado, além de informações sobre infra-estrutura.
Por fim, serão mencionadas as correlações entre o conteúdo visto em sala de aula e o que foi feito na prática, bem como as principais dificuldades encontradas durante a execução do projeto.


\section{Objetivo}
\label{sec:objetivo}

\subsection{Objetivo geral}
\label{sec:objetivoGeral}
Auxiliar no desenvolvimento dos ativos, soluções web desenvolvidas pela SEDES, participando de todas as atividades contempladas pelas etapas descritas no Modelo de Desenvolvimento de Software adotado pelo TRE-PB.  Para vistas deste relatório será utilizado como referência o projeto \imprimirtitulo. O sistema deve ser capaz de receber solicitações dos usuários e exibi-las em um painel onde o gestor deverá analisar os pedidos e montar as viagens adequando: rotas, disponibilidades dos motoristas e passageiros.

\subsection{Objetivos específicos}
\label{sec:objetivosEspecificos}
Contribuir na implementação do Sistema de Gestão de Frotas, quando possível oferecendo alternativas para a implementação mediante os conhecimentos adquiridos na graduação. Ganhar experiência com o fluxo de trabalho da uma equipe de desenvolvimento incorporando conceitos de metodologias ágeis e versionamento de código, sendo capaz de resolver conflitos de códigos, quando houver, e evitá-los. Trabalhar com reuniões diárias, planejar soluções com a modelagem do negócio especificando suas respectivas histórias de usuários, codificar utilizando as tecnologias e padrões adotados pela equipe, realizar entregas frequentes sempre buscando \textit{feedback} das \textit{releases} pelo cliente e gerar documentação explicando as funcionalidades e como utilizá-las. 

% utilizar met. ágeis, trabalhar em equipe, participar de reuniões, documentar, especificar, implementar...

\section{A Empresa}\footnote{http://www.tre-pb.jus.br/institucional/conheca-o-tre-pb/conheca-o-tre-pb}
\label{sec:empresa}
A Justiça Eleitoral é o ramo especializado do Poder Judiciário que visa garantir a lisura, a eficiência e a eficácia do processo eleitoral, contribuindo para o fortalecimento da democracia e a consolidação do Estado de Direito. Compete à Justiça Eleitoral preparar, realizar e apurar as eleições, além de administrar o Cadastro Nacional de Eleitores.
O principal objetivo da Justiça Eleitoral é o gerenciamento do processo eleitoral, através de diretrizes claras e firmes, evitando vícios, abusos e fraudes.
O Tribunal Regional Eleitoral da Paraíba - TRE-PB, órgão máximo da Justiça Eleitoral no Estado, tem como instância superior, em matéria eleitoral, o Tribunal Superior Eleitoral, sediado em Brasília - Distrito Federal. A finalidade do TRE-PB é planejar e coordenar o processo eleitoral nas eleições federais, estaduais e municipais, no âmbito do Estado da Paraíba.
Compete, também, ao Tribunal, julgar os recursos interpostos das decisões dos Juízes e Juntas Eleitorais do Estado, bem como, os processos originários e administrativos do próprio Tribunal; registrar os partidos e candidatos a cargos eletivos de Governador, Senador, Deputado Federal e Estadual, assim como, receber e analisar a prestação de contas dos mesmos, prestadas ao final de cada campanha estadual; analisar as prestações de contas anuais dos órgãos regionais dos partidos políticos; elaborar e fiscalizar o calendário estadual de propaganda eleitoral; proceder à anotação e cancelamento dos diretórios estaduais e municipais dos partidos políticos; julgar as impugnações relativas aos pedidos de registros de candidaturas e as arguições de inelegibilidade; designar os Juízes Titulares das Zonas Eleitorais do Estado da Paraíba e administrar o Cadastro de Eleitores.

\section{Descrição geral das atividades}
\label{sec:descricaoGeralAtividades}
Durante o processo foram utilizadas as metodologias Scrum para gestão e planejamento dos projetos e o Kanban para o controle de fluxo do desenvolvimento. As atividades desenvolvidas no período do estágio foram as seguintes de acordo com as fases: 
\begin{enumerate}
    \item Imersão:
    \begin{enumerate}
        \item Elicitação das histórias de usuário.
        \item Definição do escopo do produto.
        \item Criar estrutura do projeto.
        \item Modelagem de dados.
    \end{enumerate}
   \item Construção:
   \begin{enumerate}
        \item Refinar histórias de usuário.
        \item Estimar histórias.
        \item Codificação de funcionalidades.
        \item Preparar versão para homologação.
        \item Codificar ajustes.
        \item Gerar documentação.
        \item Implantar versão.
   \end{enumerate}
\end{enumerate}

\section{Organização do relatório}
\label{sec:organizacaoRelatorio}
Além desse capítulo, o relatório está dividido em outros três capítulos brevemente descritos abaixo:

\autoref{chap:embasamentoTeorico} - Embasamento Teórico: aborda as tecnologias e linguagens que foram utilizadas durante o estágio, bem como as definições necessárias para a compreensão do processo.

\autoref{chap:atividadesRealizadas} - Atividades Realizadas: apresenta o projeto, no qual as atividades do estagiário foram realizadas, descrevendo os principais conceitos e funcionalidades. Relata as atividades realizadas ao longo do período de estágio descrevendo o fluxo e o processo de desenvolvimento e detalha a implementação de algumas funcionalidades.

% TODO: revisar texto, pois as considerações finais terá outra abordagem
\autoref{chap:consideracoesFinais} – Considerações Finais: Apresenta um relato sobre as experiências adquiridas, metas e objetivos alcançados. O quão grande foi a contribuição do estágio na minha formação profissional.
